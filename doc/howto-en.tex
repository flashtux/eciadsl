% latex for the driver doc
% to compile it:
% make

% 27/09/2002 FlashCode <flashcode@flashtux.org>
% update this documentation
% translated to english by Red prophet <boristheblade@freesurf.fr>

% 07/02/2002 Benoit PAPILLAULT
% howto add an http link:
% - add the beginning of the document: \usepackage{html}
% - to add a link \htmladdnormallink{text}{URL}

% 15/02/2002 Benoit PAPILLAULT
%   change the name log by log.txt to be easier to send as an attachement
%   to an email. (.txt is automatically recognize).

% CVS $Id$
% Tag $Name$

\documentclass[a4paper,12pt]{article}
\usepackage[english]{babel}
\usepackage{html}
\usepackage[latin1]{inputenc}
\baselineskip=16pt

\title{\huge{How-to Linux driver for the ECI USB ADSL modem or Globespan based modems}}
\author{FlashCode, Bertrand Rougier, Thierry De Baere, Florent Manens, Beno�t Papillault\\
\\Translated from french to english by \htmladdnormallink{\textbf{Red prophet}}{mailto:boristheblade@freesurf.fr}}

\begin{document}
\pagenumbering{arabic}
\maketitle
\newpage
\tableofcontents
\newpage

\section{Introduction}

\subsection{Who was that document written for ?}

That document was written for all the ADSL ECI USB modem owners (or a modem equipped with a
GLOBESPAN chip) wanting to use it under Linux.
You will find the updated version of this document at this URL :
\htmladdnormallink{http://eciadsl.flashtux.org}{http://eciadsl.flashtux.org},
in the Documentation section.

The supported modem list (or supposed to be supported) along with the minimal version of
the driver needed for each modem is accessible at this URL :
\htmladdnormallink{http://eciadsl.flashtux.org/modems.php}{http://eciadsl.flashtux.org/modems.php}

\subsection{Conventions}

In the shell commands examples, the commands to type as normal user are preceded by \$
and the commands to type as root are preceded by \#.

\section{Requirements}

%    Check first that your kernel is at least 2.4.0

\subsection{Distributions}

    Distributions in which the driver is known to work properly :
    \begin{itemize}
    \item   Mandrake Linux 9.0 (kernel 2.4.19)
    \item   Mandrake Linux 8.1 (kernel 2.4.8-26mk)
    \item   Redhat Linux 8.0
    \item   Redhat Linux 7.2 (kernel 2.4.7-10)
    \item   Debian 2.2 or up (kernel 2.4.0 mimimun)
    \item   Slackware
    \item   Gentoo
    \item   ...
    \end{itemize}

    Distributions in which the driver is known to not work at all :
    \begin{itemize}
    \item   Redhat Linux 7.1 (kernel 2.4.2-2)
    \end{itemize}

    The driver is ok on MacOs X (\htmladdnormallink{http://www.hifocus.net}{http://www.hifocus.net})
    and currently being adapted for *Bsd.\\
    Contact the \htmladdnormallink{developers}{http://eciadsl.flashtux.org/support.php}
    to know more about the progress of the adaptation.

\subsection{Kernels}

To know the version of your kernel, type:
\begin{verbatim}
$ uname -r
\end{verbatim}

    Tested kernels list :

    \begin{itemize}
    \item kernel 2.4.0 : ok [usb-uhci]
    \item kernel 2.4.1 : ok [usb-uhci]
    \item kernel 2.4.2 : ok [usb-uhci]
    \item kernel 2.4.3 : ok [usb-uhci]
    \item kernel 2.4.4 : ok [usb-uhci]
    \item kernel 2.4.5 : ok [usb-uhci]
    \item kernel 2.4.6 : ok [usb-uhci]
    \item kernel 2.4.7 : ok [usb-uhci]
    \item kernel 2.4.8 : ok [usb-uhci]
    \item kernel 2.4.9 : ok [usb-uhci]
    \item kernel 2.4.10 : kernel panic process\_iso [usb-uhci] , ok [uhci]
    \item kernel 2.4.12 : ok [usb-uhci]
    \item kernel 2.4.13 : ok [usb-uhci]
    \item kernel 2.4.14 : ok [usb-uhci]
    \item kernel 2.4.15 : ok [usb-uhci]
    \item kernel 2.4.16 : ok [usb-uhci]
    \item kernel 2.4.17 : ok [usb-uhci, ohci]
    \item kernel 2.4.18 : ok [usb-uhci, ohci]
    \item kernel 2.4.19 : ok [usb-uhci, ohci]
    \item kernel 2.4.20 : ok [usb-uhci, ohci]
    \end{itemize}

Other kernels are being tested.

\subsection{pppd}

To know which version of pppd you own, type:
\begin{verbatim}
$ pppd --version
\end{verbatim}

List of tested versions:
\begin{itemize}
\item pppd 2.4.0
\item pppd 2.4.1
\end{itemize}

    This list isn't exhaustive. If you manage to connect to the internet
    with another distribution or with another kernel,
    please don't hesitate to contact us.

\section {Installation procedure (to do only once)}

Reference : the archive \textit{eciadsl-usermode-X.Y-src.tar.gz, or .rpm, or .deb} is accessible at
\htmladdnormallink{http://eciadsl.flashtux.org}{http://eciadsl.flashtux.org},
in the Download section.

\subsection{Driver compilation/installation}

    For RedHat/Mandrake package (.rpm) :

\begin{verbatim}
# rpm -i eciadsl-usermode-X.Y.rpm
\end{verbatim}

    For Debian package (.deb) :

\begin{verbatim}
# dpkg -i eciadsl-usermode-X.Y.deb
\end{verbatim}

    For sources package (.tar.gz or .tar.bz2) :

\begin{verbatim}
$ cd eciadsl-usermode-X.Y-src
$ ./configure    (only for version >= 0.6)
$ make
# make install
\end{verbatim}

    For all packages :

\begin{verbatim}
# eciconf.sh     (or eciconftxt.sh)
\end{verbatim}

    NB: eciconf.sh is a graphic configuration tool for the driver.
    It must consequently be run under X-window and requires the
    Tcl/Tk libraries.
    
    If X or Tcl/Tk isn't installed, you can run eciconftxt.sh
    (same tool as eciconf.sh, but can be run in a terminal / console).

    Sometimes it happens that the modem is on when you launch your
    computer, which is due to the dabusb module. You will need to
    remove it to install the driver and use the modem.

    There are 3 ways to remove that module :

    \textbf{1)} With eciconf.sh : unplug your modem  then in the eciconf.sh interface
    click on "Remove dabusb module".
    
    \textbf{2)} With eciconftxt.sh : choice 2, and then follow instructions.

    \textbf{3)} Type the following commands replacing VERSION by your Linux kernel version:
    \footnote{(To get your kernel version type : uname -r)}

    rm -f /lib/modules/VERSION/kernel/drivers/usb/dabusb.o.gz
    \footnote{( or rm -f /lib/modules/VERSION/kernel/drivers/usb/dabusb.o depending of the distributions)}\\
    depmod -a

    If you want to avoid a reboot :
    \begin {itemize}
    \item Type lsmod.
    \item If you see a dabusb item: type rmmod dabusb
    \end{itemize}

\subsection{ Kernel configuration (optional) }

If you want to use the 'persist' option of pppd to reconnect automatically
when disconnected, then you will have to patch your kernel or use a
kernel >= 2.4.18-pre3. You can find the patch
\textit{n\_hdlc.c.diff} in the drivers' archive of the Speedtouch modem :
\htmladdnormallink{http://speedtouch.sourceforge.net/}{http://speedtouch.sourceforge.net/}.
These are the instruction of how to use it :

\begin{verbatim}
$ cd /usr/src/linux
$ patch -p1 --dry-run < /driver's folder/n_hdlc.c.diff  ( There are 2 dashes in front of dry-run )
\end{verbatim}

If no error message is returned by the patch command, type
the following lines to make the source's patch :

\begin{verbatim}
$ cd /usr/src/linux
$ patch -p1 < /driver's folder/n_hdlc.c.diff
\end{verbatim}

Then, compile those modules for your kernel :

\begin{verbatim}
$ cd /usr/src/linux
$ make menuconfig
     Character devices --->
     [*] Non-standard serial port support
     <M> HDLC line discipline support
     [*]Unix98 PTY support
$ make clean dep modules
# make modules_install (as root)
\end{verbatim}

\subsection{pppd configuration for starting connection}

If you used eciconf.sh or eciconftxt.sh, you can go to the section 4.

\subsubsection{Script /etc/ppp/peers/adsl :}
\label{cha:peers}

That script is in the \textit{usermode-X.Y.tgz} archive and is placed with the
make install. It must look like this :

\begin{verbatim}
# 12/04/2001 Benoit PAPILLAULT <benoit.papillault@free.fr>
# 08/05/2001 Updated. Added "novj" & removed "kdebug 7"
# 07/02/2002 Replace "maxfail 0" by "maxfail 10"
# 29/04/2002 Added the option "linkname" to easily locate the running pppd
#
# This file could be rename but its place is under /etc/ppp/peers
# To connect to Internet using this configuration file, type
# pppd call adsl, where "adsl" stands for the name of this file

debug
kdebug 1
noipdefault
defaultroute
pty "/usr/local/bin/pppoeci -v 1 -vpi 8 -vci 35 -vendor 0x0915 -product 0x8000"
sync
user "adsl@adsl"
noaccomp
nopcomp
noccp
novj
holdoff 10

# This will store the pid of pppd in the first line of /var/run/ppp-eciadsl.pid
# and the interface created (like ppp0) on the second line.
linkname eciadsl

# maxfail is the number of times pppd retries to execute pppoeci after
# an error. If you put 0, pppd retries forever, filling up the process table
# and thus, making the computer unusable.
maxfail 10

usepeerdns
noauth

# If your PPP peer answer to LCP EchoReq (lcp-echo requests), you can
# use the lcp-echo-failure to detect disconnected links with:
#
# lcp-echo-interval 600
# lcp-echo-failure 10
#
# However, if your PPP peer DOES NOT answer to lcp-echo request, you MUST
# desactivate this feature with the folowing line
#
lcp-echo-interval 0

# You may need the following, but ONLY as a workaround
# mtu 1432
\end{verbatim}

\subsubsection{TO DO :}
\label{cha:TODO}

    \begin{tabular}{ll}
    - For example in France, if your ISP is Wanadoo :&
    replace :\\
        & user "\textit{adsl@adsl}"\\
    &by :\\
        &user "\textit{ fti/your\_login@fti}"\\

    - For example in France, if your ISP is Club Internet :&
    replace :\\
        &user "\textit{adsl@adsl}"\\
    &by :\\
        &user "\textit{your\_login@clubadsl1}"\\
    \end{tabular}

    HINT for some computers under Linux :

        If your firsts connections fail, check in \textit{ /var/log/messages}
        If pppd stops on a LCP message, then do the following :\\
        in \textit{ /etc/ppp/peers/adsl :}
        UN-comment the last two lines about LCP.

\subsubsection{pppd authentification script}

    Your connection password is stocked in an authentification script
    which depends on your ISP :


    \begin{tabular}{ll}
      For Wanadoo and Club Internet "Full is beautiful" :&     \textit{ /etc/ppp/chap-secrets}\\
      For Club Internet (except "Full is beautiful") :&   \textit{ /etc/ppp/pap-secrets}\\
    \end{tabular}


    These two scripts have the same syntax.
    You have to create a line as following :


    \begin{tabular}{ll}
For Wanadoo       :& \textit{ fti/your\_login@fti * your\_password *}\\
For Club Internet :& \textit{ your\_login@clubadsl1 * your\_password *}\\
    \end{tabular}


    WARNING :\\
    - the "*" are importants.\\
    - the Login entered here and the 'user' content of \textit{ /etc/ppp/peers/adsl}
    HAVE TO BE the same.

\section{Connection procedure}

\subsection{Manual (to do every time you connect to internet)}

It's very simple, the startmodem script does almost everything :

\begin{verbatim}
# /usr/local/bin/startmodem | tee log.txt
\end{verbatim}

IT DOESN'T WORK : If you can't surf the web after one of the 3
procedures, please refer to the FAQ :
\htmladdnormallink{http://eciadsl.flashtux.org/faq.php?lang=en}{http://eciadsl.flashtux.org/faq.php?lang=en}

\subsection{Automatic during Linux booting}

See this page : \htmladdnormallink{http://eciadsl.flashtux.org/faq.php?lang=en#5.0}{http://eciadsl.flashtux.org/faq.php?lang=en#5.0}

\subsection{Automatic reconnection when disconnected}

As the driver is still in beta stages, that part isn't stable yet.
Nevertheless, here are the instructions to do it. First, check that
the HDLC module isn't bugged. You can easily do it with
\textit{eci-doctor.sh}. Then, add \textbf{persist} to the
\textit{/etc/ppp/peers/adsl} file. Restart pppd, and normally,
your connection should be operational 24/7.

For other deconnections (USB problems for example), you can download
a reconnection script, available here :
\htmladdnormallink{http://eciadsl.flashtux.org/download.php?lang=en&view=reco}{http://eciadsl.flashtux.org/download.php?lang=en&view=reco}

\section{Contacts}

\subsection{Support}

All the ways to contact the support are here :
\htmladdnormallink{http://eciadsl.flashtux.org/support.php?lang=en}{http://eciadsl.flashtux.org/support.php?lang=en}

\subsection{Mailing list}

For any question, subscribe to this project's mailing list
sending a blank mail to :
\htmladdnormallink{\textbf{eci-request@ml.free.fr}}{mailto:eci-request@ml.free.fr?subject=subscribe}
with \textbf{subscribe} in the subject field, then post your remarks or suggestions to :
\htmladdnormallink{\textbf{eci@ml.free.fr}}{mailto:eci@ml.free.fr}.

\subsection{IRC}

You can connect to the IRC server
\htmladdnormallink{OpenProjects.Net}{http://www.openprojects.net/} on
the \textbf{\#eci} channel on the server :
\htmladdnormallink{\textbf{irc.openprojects.net}}{irc://irc.openprojects.net/eci}
for example.

\end{document}
