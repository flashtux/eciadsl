\documentclass[a4paper,12pt]{article}
\usepackage[colorlinks,urlcolor=blue,linkcolor=blue]{hyperref}

\begin{document}

\title{\huge{Design of a usermode driver for the ECI ADSL USB modem}}
\author{Beno�t PAPILLAULT, \href{mailto:benoit.papillault@free.fr}{benoit.papillault@free.fr}}

\maketitle
\newpage
\tableofcontents
\newpage

\section{Introduction}

This document was started March 2003, 04. There is a need of a totally
new design for the usermode driver, since :

\begin{itemize}
\item the current driver is based on a lot of workaround in the USB
filesystem. This can result to buggy driver anytime someone make a
modification.
\item the current driver is using ISO USB transfers which seems to be
the reason of most disconnections.
\end{itemize}

The new design has to be modular.

\section{Basic design}

The role of the current driver is to make a link between the USB
filesystem where the ECI ADSL USB modem is accessible and the PPP
layer. This is done through the binary \textit{pppoeci} which is
launch by \textit{pppd}. \textit{pppd} and \textit{pppoeci}
communicate through stdin and stdout.

But currently, \textit{pppoeci} is just a big program, not a modular
program.

\section{Modular design}

The aim of any modular design is to separate functionality:

\begin{itemize}
\item USB access should be handled by a separate library, that should
be ported to several Unices, like Linux and BSD (OpenBSD, NetBSD,
FreeBSD).
\item PPP access should be done through a layer of module, one module
handling only its own layer.
\item Modem access should be done by a separate module. In this way,
we can hope to handle several modem using different protocols (like
the ECI ADSL USB modem and the Speedtouch modem).
\end{itemize}

\section{USB library}

To avoid having to deal with OS differences, this library should be
complete. To be used in different projects, this library should be packaged
separately. The package name will be \textbf{pusb}.

\subsection{USB library : initialisation and destruction}

To initialize the library:

\begin{verbatim}
pusb\_init();
\end{verbatim}

To free all the memory internally allocated by the library:

\begin{verbatim}
pusb\_done();
\end{verbatim}

\subsection{USB library : enumerating devices}

The first step is to enumerate devices to find the one you want to handle:

\begin{verbatim}
int pusb\_enumerate\_first();
\end{verbatim}

\begin{verbatim}
int pusb\_enumerate\_next();
\end{verbatim}

\subsection{Opening and closing USB devices}

\begin{verbatim}
pusb\_device\_t pusb\_open(const char *path);

int pusb\_close(pusb\_device\_t dev);
\end{verbatim}

\subsection{USB library : sending and receiving data}

A USB device can handle 4 types of transfers :
\begin{itemize}
\item CTRL transfers
\item BULK transfers
\item INT  transfers
\item ISO tranfers
\end{itemize}

\subsubsection{CTRL transfers}

int pusb\_control\_msg(pusb\_device\_t dev,
		     int request\_type, int request,
		     int value, int index, 
		     unsigned char *buf, int size, int timeout);

\subsubsection{BULK transfers}

\begin{verbatim}
int pusb\_endpoint\_read(pusb\_endpoint\_t ep, 
		       unsigned char *buf, int size, int timeout);

int pusb\_endpoint\_write(pusb\_endpoint\_t ep, 
			const unsigned char *buf, int size, int timeout);
\end{verbatim}

\subsubsection{INT transfers}

\subsubsection{ISO transfers}

\subsection{USB library : how to use it}

At the top of your source code, just add:

\begin{verbatim}
#include <pusb.h>
\end{verbatim}

In your \textbf{Makefile}, add the following:

\begin{verbatim}
LDLIBS += -lpusb
\end{verbatim}

\section{modem library}

\section{layer module}

\end{document}

