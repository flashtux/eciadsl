\documentclass[a4paper,12pt]{article}
\usepackage[latin1]{inputenc}
\usepackage[colorlinks,urlcolor=blue,linkcolor=blue]{hyperref}
\usepackage{html}

\begin{document}

\title{\huge{Design of a usermode driver for the ECI ADSL USB modem}}
\author{Beno�t PAPILLAULT,
\htmladdnormallink{benoit.papillault@free.fr}{mailto:benoit.papillault@free.fr}}

\maketitle
\newpage
\tableofcontents
\newpage

\section*{About this document}

The \textbf{eciadsl} project was started in year 2001 to produce a
Linux driver for the ECI ADSL USB modem. One basic choice was to
produce a usermode driver so that installation will not need a kernel
compilation, which is even tedious for beginner. That also mean that
we have to use a usermode library to access USB devices. This document
will try to describe a possible design for the usermode driver.

This document is part of the \textbf{eciadsl} project documentation,
available on the web at
\htmladdnormallink{http://eciadsl.flashtux.org/}{http://eciadsl.flashtux.org}.
The original source file is \textbf{usermode-design.tex}. Other
formats have been generated from this source file.

\section{Introduction}

This document was started March 2003, 04. There is a need of a totally
new design for the usermode driver, since:

\begin{itemize}

\item On Linux, the current driver is based on a lot of workaround in
the USB filesystem. This can result in buggy driver anytime if someone
make a modification in the usermode driver source code.

\item the current driver is using ISO USB transfers which seems to be
the reason of most disconnections.

\end{itemize}

The new design has to be modular. That means that source code for the
PPP layer and the USB layer should be in different source files for
instance.

As far as USB libraries are concerned, the only Existing library
\textbf{libusb} do not support ISO transfer. That means it is not
usable in our project. See
\htmladdnormallink{http://libusb.sourceforge.net/}
{http://libusb.sourceforge.net/} for further information on the libusb
project, maintained by Johannes Erdfelt.

\section{Basic design}

The role of the current driver is to make a link between the USB
filesystem where the ECI ADSL USB modem is accessible and the PPP
layer. This is done through the binary \textit{pppoeci} which is
launch by \textit{pppd}. \textit{pppd} and \textit{pppoeci}
communicate through stdin and stdout.

But currently, \textit{pppoeci} is just a big program, not a modular
program.

\section{Modular design}

The aim of any modular design is to separate functionality:

\begin{itemize}
\item USB access should be handled by a separate library, that should
be ported to several Unices, like Linux and BSD (OpenBSD, NetBSD,
FreeBSD).
\item PPP access should be done through a layer of module, one module
handling only its own layer.
\item Modem access should be done by a separate module. In this way,
we can hope to handle several modems using different protocols (like
the ECI ADSL USB modem and the Speedtouch modem).
\end{itemize}

\section{USB library}

To avoid having to deal with differences in various OS, this library
should be complete. That means that everything which is possible to do
with USB should be possible using this library. To be used in
different projects, this library should be packaged separately. The
package name will be \textbf{pusb}. The library should take into
account the fact that a USB device may be added or removed to the
system at any time.

\vspace{0.3cm}

For further information on the USB protocol, you can browse
documentation at
\htmladdnormallink{http://www.usb.org/developers/docs}
{http://www.usb.org/developers/docs}

\vspace{0.3cm}

Target OS:
\begin{itemize}
\item Linux (done for the speedtouch modem)
\item OpenBSD, NetBSD, FreeBSD (done for the speedtouch modem)
\item QNX

  Documentation about USB support under QNX is available. Everything
  is already done in usermode : \\
  \htmladdnormallink{http://www.qnx.com/developer/docs/momentics621\_docs/ddk\_en/usb/about.html}
  {http://www.qnx.com/developer/docs/momentics621_docs/ddk_en/usb/about.html}.

\item Mac OS
\item Mac OS X
\item BeOS
\item Solaris
\end{itemize}

Pour chaque OS : 

-> document d'installation + fichiers � t�l�charger
-> installation d'un env. dev. (emacs, cvs, gcc, ssh client/server)
-> documentation USB (lib, ddk, .h)
-> package (rpm, deb, autres)

\subsection{USB library: initialization and destruction}

Initialisation and destruction functions in a library are always a
pain if your library is used by several other libraries that are used
at the same time in your project. In this case, that means your
initialisation function can be called multiple times and if we are in
a multithreaded programs, we need to use mutexes to avoid
bugs. Currently, we don't need such functions. So, we will not
implement them and will not use them. However, we keep their
definitions here, in case it is later needed.

To initialize the library:

\begin{verbatim}
void pusb_init();
\end{verbatim}

Should return a bool indicating the success of the init ?


To free all the memory internally allocated by the library:

\begin{verbatim}
void pusb_done();
\end{verbatim}

\subsection{USB library: enumerating devices}

The first step is to enumerate devices to find the one you want to handle:

\begin{verbatim}
int pusb_enum_init(pusb_enum_t *ref);
\end{verbatim}

This function returns \textbf{TRUE} if there are some USB devices in
the system and in this case initializes the value pointed by
\textit{ref}. If there is no USB device in the system, the function
returns \textbf{FALSE}.

\begin{verbatim}
int pusb_enum_next(pusb_enum_t *ref);
\end{verbatim}

This function is used to enumerate the next USB device in the list. If
there is such a device, the function returns \textbf{TRUE} and the
value pointed by \textit{ref} is initialized with the next USB
device. If there is no next USB device, the function returns
\textbf{FALSE}.

\begin{verbatim}
void pusb_enum_done(pusb_enum_t * ref);
\end{verbatim}

This function free the memory allocated for \textit{ref} by
\textit{pusb\_enum\_init} or \textit{pusb\_enum\_next}.

The typical use of such functions is like in the following code
sample:
\begin{verbatim}
    pusb_enum_t ref;

    pusb_enum_init(&ref);

    while (pusb_enum_next(&ref)
    {
        /* do something with ref even break; */
    }

    pusb_enum_done(&rref);
\end{verbatim}

\subsubsection*{Note:}

\begin{itemize}

\item The order in which USB devices are reported may vary from call
to call and from OS to OS.

\end{itemize}

\subsubsection*{Questions:}

\begin{itemize}
\item
What happens if a USB device is added or removed while
\textit{pusb\_enum\_next}() is called?

The result is undefined. However, if a device is present when
\textit{pusb\_enum\_init}() is called and is still present when
\textit{pusb\_enum\_done}() is called, the device will be
listed. Devices added or removed between those two calls may or may
not get listed, depending on the implementation of the library.

\item
Why not creating a \textit{pusb\_open\_device}(int vid, int pid) which
return the first device matching vid/pid?

Such function could ease the developpement of simple device driver,
however this will have the main drawback that the result is undefined
if more than once device with the same pid and vid exist in the
system.

\end{itemize}

\begin{verbatim}
int pusb_get_device_class(pusb_enum_t * ref); // 8 bits

int pusb_get_device_subclass(pusb_enum_t * ref); // 8 bits

int pusb_get_device_protocol(pusb_enum_t * ref); // 8 bits

int pusb_get_device_vendor(pusb_enum_t * ref); // 16 bits

int pusb_get_device_product(pusb_enum_t * ref); // 16 bits

int pusb_get_device_version(pusb_enum_t * ref); // 16 bits
\end{verbatim}

All those functions gives some information about the device. 

\subsection{USB library: opening and closing a device}

\begin{verbatim}
pusb_device_t pusb_device_open(pusb_enum_t * ref);

int pusb_device_close(pusb_device_t dev);
\end{verbatim}

\textit{pusb\_open}() is used to open a USB device. If the function
fails, the return value is NULL. \textit{pusb\_close}() is used to close
the device once it has been successfully opened with
\textit{pusb\_open}(). What happens to endpoint that has been opened
using this device if the device is closed?

\subsection{USB library: configuring the device}

A USB device is made up of one or several configurations. Only one
configuration can be set at any given time.  The differences between
each configurations might be the electrical power needed. Next, each
configuration can have several alternate settings. Each alternate
settins determine the set of endpoints that will be presented to the
application. Once again, only one alternate settings can be selected
at any given time. Now, each endpoints of the active configuration and
alternate setting can use used together at the same time.

For instance, here is a list of all the possible configuration of the ECI ADSL USB modem:
\begin{verbatim}
Device: VendorID=0x0915 ProductID=0x8000 Class=ff/ff/ff, 1 configuration(s)
Manufacturer: GlobeSpan Inc. Product: USB-ADSL Modem SN: FFFFFF
  configuration 1, 1 interface(s) []
    interface 0 alt 0 class ff/ff/ff, 3 endpoint(s) []
      endpoint 0x82 [Bulk] 64 bytes 0 ms
      endpoint 0x02 [Bulk] 64 bytes 0 ms
      endpoint 0x86 [Intr] 64 bytes 3 ms
    interface 0 alt 1 class ff/ff/ff, 3 endpoint(s) []
      endpoint 0x88 [Isoc] 1008 bytes 1 ms
      endpoint 0x02 [Bulk] 64 bytes 0 ms
      endpoint 0x86 [Intr] 64 bytes 3 ms
    interface 0 alt 2 class ff/ff/ff, 3 endpoint(s) []
      endpoint 0x88 [Isoc] 912 bytes 1 ms
      endpoint 0x02 [Bulk] 64 bytes 0 ms
      endpoint 0x86 [Intr] 64 bytes 3 ms
    interface 0 alt 3 class ff/ff/ff, 3 endpoint(s) []
      endpoint 0x88 [Isoc] 736 bytes 1 ms
      endpoint 0x02 [Bulk] 64 bytes 0 ms
      endpoint 0x86 [Intr] 64 bytes 3 ms
    interface 0 alt 4 class ff/ff/ff, 3 endpoint(s) []
      endpoint 0x88 [Isoc] 448 bytes 1 ms
      endpoint 0x02 [Bulk] 64 bytes 0 ms
      endpoint 0x86 [Intr] 64 bytes 3 ms
    interface 0 alt 5 class ff/ff/ff, 3 endpoint(s) []
      endpoint 0x88 [Isoc] 240 bytes 1 ms
      endpoint 0x02 [Bulk] 64 bytes 0 ms
      endpoint 0x86 [Intr] 64 bytes 3 ms
    interface 0 alt 6 class ff/ff/ff, 3 endpoint(s) []
      endpoint 0x88 [Isoc] 96 bytes 1 ms
      endpoint 0x02 [Bulk] 64 bytes 0 ms
      endpoint 0x86 [Intr] 64 bytes 3 ms
\end{verbatim}

We need to add some function like:

\begin{verbatim}

int pusb_device_set_configuration(pusb_device_t dev, int config);

int pusb_device_set_interface(pusb_device_t dev, int interface, int alternate);

\end{verbatim}

\subsection{USB library: opening and closing an endpoint}

\begin{verbatim}
pusb_endpoint_t pusb_endpoint_open(pusb_device_t dev, int epnum);

int pusb_endpoint_close(pusb_endpoint_t ep)
\end{verbatim}

\textit{pusb\_endpoint\_open} open the endpoint whose number if
\textit{epnum} from the device specified by \textit{dev}. If an error
occurs, NULL is returned. \textit{pusb\_endpoint\_close} close a
endpoint that was sucessfully opened with \textit{pusb\_endpoint\_open}.

\subsubsection*{Note:}

\textit{epnum} is a 8 bit value defining a USB endpoint. USB endpoints
are either read-only or write-only. You cannot read and write on the
same endpoints.

\subsection{USB library: sending and receiving data}

A USB device can handle 4 types of transfers:
\begin{itemize}
\item Control transfers
\item Bulk transfers
\item Interrupt transfers
\item Isochronous tranfers
\end{itemize}

For all kind of transfer, you can set a timeout value.

\begin{verbatim}
int pusb_endpoint_set_timeout(pusb_endpoint_t ep, int timeout);

int pusb_endpoint_get_timeout(pusb_endpoint_t ep);
\end{verbatim}

If the timeout is zero, which is the \textbf{default}, transfer will
be blocking. That means that each read/write functions will wait for
ever for its result. This work like the usual read and write
functions.

If the timeout is non-zero, it defines a delay in milliseconds. The
read/write functions will wait for the specified duration for the
result, if no result comes, an error is returned.

\subsubsection{Control transfers}

Control transfers are used for configuring a USB device. Data are
delivered as best effort. No bandwidth can be reserved for control
tranfers.

\begin{verbatim}
int pusb_control_msg(pusb_device_t dev,
		     int request_type, int request,
		     int value, int index, 
		     unsigned char *buf, int size);
\end{verbatim}

Directions of the transfer is determined by bit 7 of
\textit{request\_type}: a value of 0 indicates a WRITE and a value of
1 indicates a READ.

Meaning of request\_type ? request? value ? index ? buf ? size ?
Range for all of those values?

\textit{size} is the number of bytes from the buffer \textit{buf} that
will be transfered. The return value, if positive, indicates the
number of bytes that was transfered.

\subsubsection{Bulk transfers}

Bulk transfers have the following properties:

\begin{itemize}
\item Access to the USB on a bandwidth-available basis
\item Retry of transfers, in the case of occasional delivery failure
due to errors on the bus
\item Guaranteed delivery of data but no guarantee of bandwidth or latency
\end{itemize}

The following functions should be used :

\begin{verbatim}
int pusb_endpoint_bulk_read(pusb_endpoint_t ep,
                            unsigned char *buf, int size);

int pusb_endpoint_bulk_write(pusb_endpoint_t ep,
                             const unsigned char *buf, int size);
\end{verbatim}

\subsubsection{Interrupt transfers}

Interrupt transfers have the following properties:

\begin{itemize}
\item Guaranteed maximum service period for the pipe
\item Retry of transfer attempts at the next period, in the case of
occasional delivery failure due to error on the bus
\end{itemize}

The following functions should be used :

\begin{verbatim}
int pusb_endpoint_int_read(pusb_endpoint_t ep,
                           unsigned char *buf, int size);

int pusb_endpoint_int_write(pusb_endpoint_t ep,
                            const unsigned char *buf, int size);
\end{verbatim}

Questions: 
\begin{itemize}
\item What call is used to read from/write to Interrupt endpoint?
\item Blocking/non-blocking calls?
\item How to stop reading from/writing to an Interrupt endpoint?
\end{itemize}

\subsubsection{Isochronous transfers}

Isochronous transfers have the following properties:

\begin{itemize}
\item Guaranteed access to USB bandwidth with bounded latency
\item Guaranteed constant data rate through the pipe as long as data
is provided to the pipe
\item In the case of a delivery failure due to error, no retrying of
the attempt to deliver the data
\end{itemize}

The following functions should be used :

\begin{verbatim}
int pusb_endpoint_iso_read(pusb_endpoint_t ep,
                           unsigned char *buf, int size);

int pusb_endpoint_iso_write(pusb_endpoint_t ep,
                            const unsigned char *buf, int size);
\end{verbatim}

Questions:
\begin{itemize}
\item What calls to read from/write to an Isochronous endpoint?
\end{itemize}

\subsection{USB library: how to use it}

At the top of your source code, just add:

\begin{verbatim}
#include <pusb.h>
\end{verbatim}

If you are using a \textbf{Makefile}, add the following:

\begin{verbatim}
CFLAGS  += -I/usr/local/include
LDFLAGS += -L/usr/local/lib
LDLIBS  += -lpusb
\end{verbatim}

If you are using a \textbf{configure.ac} file, add the following:

\begin{verbatim}
AM_PATH_PUSB(,,)
\end{verbatim}

You also need to copy the content of \textbf{pusb.m4} into your
\textbf{configure.ac} file before using \textit{AM\_PATH\_PUSB}.

Also add the following to \textbf{Makefile.in} :

\begin{verbatim}
CFLAGS += @PUSB_CFLAGS@
LDLIBS += @PUSB_LIBS@
\end{verbatim}

\section{Modem library}

The goal is to create a library that will not be modem specific, as
far as possible. In fact, we can sum up the behavior of most ADSL
modems by the followings steps:

\begin{itemize}
\item some ADSL modems need a firmware and there is a specific
  procedure to follow to load such firmware once, ie each time it's
  plugged on the USB bus.

\item all ADSL modems needs a specific procedure to synchronize with
  the ADSL line. This involves some modulation detection and settings
  (G.DMT, T1.143, ...).

\item all ADSL modems have a way to send and receive ATM cells. This
  layer is the common denominator to all ADSL modems.

\end{itemize}

\section{Layer module}

The goal is to be able to quickly build up a protocol layer using
basic module for each layer. Let's examine the protocol layer of some
ADSL connections.

\subsection{ADSL connections using PPPoA}

PPPoA is a protocol describe in \htmladdnormallink{RFC
2364}{http://www.ietf.org/rfc/rfc2364.txt}. It describes a way to
encapsulate PPP packets into AAL5 (ATM Adaptation Layer 5). There are two types of encapsulation to do this:

\begin{itemize}
\item VC-multiplexed PPP (VC-mux)
\item LLC encapsulated PPP (LLC-encap)
\end{itemize}

\subsubsection{PPPoA: VC-mux}

This is the simplest encapsulation since one PPP packet is directly
the payload of an AAL5 packet. In the current driver, this
encapsulation is called \textbf{VCM\_RFC2364}.

\subsubsection{PPPoA: LLC-encap}

In this encapsulation type, the PPP packet is preceded by the byte
sequence: 0xfe 0xfe 0x03 0xcf and put into the payload of an AAL5
packet. In the current driver, this encapsulation is called
\textbf{LLC\_RFC2364}.

\subsection{ADSL connections using PPPoE}

\subsubsection{PPPoE: LLC-encap Routed Protocol non-ISO}

In this encapsulation type, the IP packet is preceded by the byte
sequence: 0xaa 0xaa 0x03 0x00 0x00 0x00 0x08 0x00 and put into the
payload of an AAL5 packet. In the current driver, this encapsulation
is called \textbf{LLC\_RFC1483\_ROUTED\_IP}.

\subsubsection{PPPoE: LLC-encap Bridged Ethernet/802.3}

In this encapsulation type, the Ethernet packet is preceded by the
byte sequence: 0xaa 0xaa 0x03 0x00 0x80 0xc2 0x00 0x07 and put into
the payload of an AAL5 packet. In the current driver, this
encapsulation is called
\textbf{LLC\_SNAP\_RFC1483\_BRIDGED\_ETH\_NO\_FCS}.

\subsubsection{PPPoE: VC-mux Routed Protocol}

In this encapsulation type, the IP packet is put into the payload of
an AAL5 packet. In the current driver, this encapsulation is called
\textbf{VCM\_RFC1483\_ROUTED\_IP}.

\subsubsection{PPPoE: VC-mux Bridged Ethernet/802.3}

In this encapsulation type, the Ethernet packet is preceded by the
byte sequence: 0x00 0x00 and put into the payload of an AAL5
packet. In the current driver, this encapsulation is called
\textbf{VCM\_RFC\_1483\_BRIDGED\_ETH}.

\subsection{basic layers}

From the above examples, we notice that there are some common
layers. Thus, our goal is to provide basic module for each layer.

\begin{itemize}

\item PPP

\item PPPoA

\item PPPoE

\end{itemize}

\end{document}

% email mutley : eau@phear.org
% email hb     : bhuisgen@wanadoo.fr
