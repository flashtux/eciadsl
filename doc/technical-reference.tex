\documentclass[a4paper,12pt]{article}
% \usepackage{babel}
\usepackage{html}
% \usepackage[latin1]{inputenc}
% \baselineskip=16pt

\title{\huge{Technical Reference of the ECI ADSL modem}}
\author{Benoit PAPILLAULT}

\begin{document}
\pagenumbering{arabic}
\maketitle
\newpage
\tableofcontents
\newpage

\section{Introduction}

This documents aims at documenting the hardware part of the ECI ADSL
modem, without any consideration of operating system in mind.

The basic design of this modem is made of 2 chips:
\begin{itemize}
\item a Cypress EZ-USB chip which is responsible for the USB interface
connected to the computer (via the USB bus).
\item a GlobeSpan chip which is responsible of handling the ADSL part
\end{itemize}

We will try to document those chips and some others. This documents
will also list some useful links found on the Internet

\section{Cypress EZ-USB}

The exact reference is :
\begin{verbatim}
     CYPRESS
  EZ-USB (TM)
  AN2131QC
    0103
ENY2984 KOREA
\end{verbatim}

It has the following features:
\begin{itemize}
\item USB Bus Specification version 1.1
\item UTOPIA Level I v1.0 and Level II v2.01 support
\item +3.3-volt, 80-pin PQFP
\end{itemize}

Official links are :
\begin{itemize}
\item Summary :
\htmladdnormallink{http://www.cypress.com/products/datasheet.cfm?partnum=AN2131QC}{http://www.cypress.com/products/datasheet.cfm?partnum=AN2131QC}
\item \htmladdnormallink{EZ-USB Technical Reference Manual
v1.9}{http://www.cypress.com/cfuploads/support/developer\_kits/EF9F4266-6D9F-43B3-97CC2EDFBEC57C68\_doc\_1.pdf},
1558199 bytes (pdf).
\item \htmladdnormallink{EZ-USB FX Technical Reference Manual
v1.2}{http://www.cypress.com/cfuploads/support/developer\_kits/FF8EFAB3-2E98-4448-92418D0EA786766D\_doc\_1.pdf},
5462001 bytes (pdf).
\item \htmladdnormallink{EZ-USB FX2 Technical Reference Manual
v2.1}{http://www.cypress.com/cfuploads/support/developer\_kits/D264112D-9052-4BA8-9DBB23543A0A1A9B\_doc\_1.zip},
2440894 bytes (pdf/zip).
\item EZ-USB IO Ports (PDF) :
\htmladdnormallink{http://www.cypress.com/cfuploads/support/app\_notes/EZUSB\_IO\_Ports.pdf}{http://www.cypress.com/cfuploads/support/app\_notes/EZUSB\_IO\_Ports.pdf}
\item Software SPI Implementation on EZ-USB (PDF) :
\htmladdnormallink{http://www.cypress.com/cfuploads/support/app\_notes/Software\_SPI\_Implementation.pdf}{http://www.cypress.com/cfuploads/support/app\_notes/Software\_SPI\_Implementation.pdf}
\end{itemize}

Basically, it's an enhanced 8051 with 8K of memory, running at 24Mhz.

Links :
\begin{itemize}
\item 8051 Tutorial :
\htmladdnormallink{http://8052.com/tut8051.phtml}{http://8052.com/tut8051.phtml}
\end{itemize}

\section{Globespan GS7070 ADSL DMT Modem and ATM Framer}

The exact reference is :
\begin{verbatim}
GS7070-174-008D A2
\end{verbatim}

Official link to the GlobeSpan site : \htmladdnormallink{http://www.globespan.net/products/ticpe.html}{http://www.globespan.net/products/ticpe.html}

From the link \htmladdnormallink{http://www.asi-tec.com/php/sklep.php?go=opisprod\&do=spec\_dsl100u\_pl}{http://www.asi-tec.com/php/sklep.php?go=opisprod\&do=spec\_dsl100u\_pl}, we learn :
\begin{itemize}
\item ANSI T1.413 Issue 2 category 1 \& 2 standard DMT modem with
embedded ATM Framer
\item Standard Utopia Level I and Level II ATM Interfaces
\item DMT Modulation up to maximum of 256 tones (14 bits)
\item ADSL/ATM cell-specific Framing and Deframing
\item Performs DMT Modulation, Demodulation, Reed-Solomon Encoding,
Bit Interleaving, and 4D Trellis Cod
\end{itemize}



\section{GlobeSpan GT3180 ADSL ANALOG FRONT-END}

The exact reference is :
\begin{verbatim}
GT3180-01-A
   0041S
 92802741
\end{verbatim}

From the same link as above, we learn:
\begin{itemize}
\item Integrated Analog Front End (AFE) for ADSL
\item Differential Analog Input/Output
\item 3.3-volt \& 5-volt supplies, 100-pin TQFP
\end{itemize}

\section{Cypress \#2}

The exact reference (U602) is :
\begin{verbatim}
CY7C1399-15ZC
0110 F 04
609369
\end{verbatim}

CY1399-15Z 32K x 8 3.3V Static RAM 15ns.
\begin{itemize}

\item
\htmladdnormallink{http://www.cypress.com/products/datasheet.cfm?partnum=CY7C1399\%2D15ZCTQ}{http://www.cypress.com/products/datasheet.cfm?partnum=CY7C1399\%2D15ZCTQ}
\item
\htmladdnormallink{http://www.mespek.com/p\_sram2001compact.pdf}{http://www.mespek.com/p\_sram2001compact.pdf}
\end{itemize}

\section{Other}

The exact reference is :
\begin{verbatim}
LCX
573
PPXX
\end{verbatim}

\section{Links}

\subsection{8051 tools}

Here are some links:
\begin{itemize}
\item 8051 Disassembler :
\htmladdnormallink{http://wuarchive.wustl.edu/aminet/dirs/aminet/dev/cross/8051.lha}{http://wuarchive.wustl.edu/aminet/dirs/aminet/dev/cross/8051.lha}

To install this disassembler, use the following procedure :
\begin{verbatim}
$ lha x ../ftp/ezusb/8051.lha
$ cd 8051/DisAsm/
$ gcc -o d51 Dis8051.c
\end{verbatim}

To use it, don't forget to create a file \textit{entries} in the
current directory, containing all entry points (in our case,
0000). And next, using an Intel Hex Record file as input:
\begin{verbatim}
$ d51 < firm.hex > firm.s
\end{verbatim}

\item 8051 C Compiler :
\htmladdnormallink{http://sdcc.sourceforge.net/}{http://sdcc.sourceforge.net/}

To install sdcc from the source package, get sdcc-XXX.tar.gz from the
above link, unpack the archive and use the usual magic process :
\begin{verbatim}
$ cd sdcc
$ ./configure
$ make
$ su
# make install
\end{verbatim}

\end{itemize}

\end{document}


