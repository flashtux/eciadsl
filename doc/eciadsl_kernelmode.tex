% latex for the driver doc
% to compile it:
% make

% 12 June 2002 jeanseb


% CVS $Id$
% Tag $Name$

\documentclass[a4paper,12pt]{article}
\usepackage[francais]{babel}
\usepackage{html}
\usepackage[latin1]{inputenc}
\baselineskip=16pt

\title{\huge{How-to pour le driver Linux du modem ECI USB - Version noyaux}}
\author{Jean-S�bastien Valette jeanseb.valette@free.fr}

\begin{document}
\pagenumbering{arabic}
\maketitle
\newpage
\tableofcontents
\newpage


\section{Introduction}

\subsection{A qui s'adresse ce document ?}

Ce document s'adresse � tous les possesseurs du modem ADSL ECI USB
voulant l'utiliser sous Linux et ayant une bonne connaissance de la configuration et de l'installation de linux. Vous trouverez la derni�re version de
ce documents sur le site :
\htmladdnormallink{http://eciadsl.sourceforge.net/}{http://eciadsl.sourceforge.net/},
section Documentation.

\subsection{Conventions}

Lors des exemples de commandes � taper au niveau du shell, les
commandes � ex�cuter en tant qu'utilisateur normal sont pr�c�d�es de \$
et les commandes � ex�cuter en root sont pr�c�d�es de \#.

\section{Configuration requise}

%    Assurez vous d'abord que vous poss�dez un noyau Linux 
%    avec le support de pppoatm et que vous avez installer la 
%    librairie atm.

\subsection{Distributions}

    Distributions o� le driver fonctionne : 
    \begin{itemize}
    \item   debian woody (noyau 2.4.18 - recompil�)
    \end{itemize}
    
    Distributions o� le driver ne fonctionne pas :
    \begin{itemize}
	Support d'ipfilter activer.
    \end{itemize}

\subsection{Noyaux}

Pour obtenir la version de votre noyau:
\begin{verbatim}
$ uname -r
\end{verbatim}

    Liste des noyaux test�s (en utilisant la version CVS dont le tag est ``release-test'') :

    \begin{itemize}
    \item kernel 2.4.18 : ok [usb-uhci]
    \end{itemize}

D'autres noyaux sont en cours de test.

\subsection{pppd}

Pour obtenir la version de pppd:
\begin{verbatim}
$ pppd --version
\end{verbatim}

Liste des versions test�es:
\begin{itemize}
\item pppd 2.4.2 beta
\end{itemize}

    Cette liste n'est pas exhaustive. Si vous parvenez � vous connecter
    avec une autre distribution ou avec un noyau plus ancien,
    n'h�sitez pas � nous contacter.

\section {Recuperation des sources du driver et de pppd}

\subsection{Recuperation du driver}

    Connectez vous en tant que utilsatuer puis tapez :

\begin{verbatim}
$ mkdir ecidasl   
$ cd eciadsl
$ cvs -d :pserver:cvs@pserver.samba.org:/cvsroot login
\end{verbatim}
Taper cvs comme mot de passe.
ensuite, toujours dans le shell.
\begin{verbatim}
$ cvs -z5 -d :pserver:cvs@pserver.samba.org:/cvsroot co ppp
$ cd ppp/
$ make
$ cd ..
$ cvs -d:pserver:anonymous@cvs.eciadsl.sourceforge.net:/cvsroot/eciadsl login 
$ cvs -z3 -d:pserver:anonymous@cvs.eciadsl.sourceforge.net:/cvsroot/eciadsl co kernelmode 
\end{verbatim}
    Lors du prompt pour le mot de passe valider simplement.

    Vous avez maintenant r�cuperer les sources du drivber et de pppd.
\subsection{ Compilation et installation  }


\begin{verbatim}
$ cd kernelmode 
$ make
$ cd ../ppp/pppd
$ make
\end{verbatim}



    
\subsection{Configuration de pppd pour l'ouverture de la connexion}

\subsubsection{Script /etc/ppp/peers/test :}
\label{cha:peers}

Il vest dans le r�pertoire \textit{kernelmode} et doit etre recopier avnt de lancer pppd.
Il doit ressembler � �a :

\begin{verbatim}
# 12/04/2001 Benoit PAPILLAULT <benoit.papillault@free.fr>
# 08/05/2001 Updated. Added "novj" & removed "kdebug 7"
#
# This file could be rename but its place is under /etc/ppp/peers
# To connect to Internet using this configuration file
# pppd call adsl, where "adsl" stands for the name of this file

debug
kdebug 1
noipdefault
noauth
defaultroute
plugin /home/jeanseb/driver/kernelmode/pppoatm.so
8.35
#sync
user "fti/@fti"
#noaccomp
#nopcomp
#noccp
#novj
holdoff 1
#persist
maxfail 0
usepeerdns
#lcp-echo-interval 600
#lcp-echo-failure 5

\end{verbatim}

\subsubsection{A FAIRE avant de tester}
\label{cha:faire}

    \begin{tabular}{ll}
    - si vous �tes abonn� wanadoo :&
    remplacer :\\
        & user "\textit{adsl@adsl}"\\
    &par :\\
        &user "\textit{ fti/votre\_login@fti}"\\

    - si vous �tes abonn� club-internet :&
    remplacer :\\
        &user "\textit{adsl@adsl}"\\
    &par :\\
        &user "\textit{votre\_login@clubadsl1}"\\
    \end{tabular}
    
    \begin{tabular}{ll}
    - n'oubliait pas de modifier le path pour le plugin
    \end{tabular}
    ASTUCE sur certaines machines Linux :
    
        Si vos premi�res connexions �chouent, voyez dans \textit{ /var/log/messages}
        Si pppd s'arr�te sur un message contenant LCP, alors faites cela :\\
        dans \textit{ /etc/ppp/peers/adsl :}
        DE-commentez les 2 derni�res lignes parlant de LCP.



\subsubsection{Parametrage des mot de passe (si pas deja fait pour l'usermode)}

    Votre mot de passe sera stock� dans un script d'authentification
    qui d�pend de l'op�rateur internet que vous utilisez :
    
    
    \begin{tabular}{ll}
      Pour wanadoo :&     \textit{ /etc/ppp/chap-secrets}\\
      Pour club-internet :&   \textit{ /etc/ppp/pap-secrets}\\
    \end{tabular}
    
    
    Les 2 scripts ont exactement la m�me syntaxe.
    Vous devez cr�er un ligne comme suit :
    
    
    \begin{tabular}{ll}
Pour wanadoo       :& \textit{ fti/votre\_login@fti * votre\_password *}\\
Pour club-internet :& \textit{ votre\_login@clubadsl1 * votre\_password *}\\
    \end{tabular}
    
    
    ATTENTION :\\
    - les "*" sont importants.\\
    - le premier champ de la ligne DOIT �tre �gal
      au contenu de la ligne user du fichier \textit{ /etc/ppp/peers/adsl}

\section{Proc�dure de test}


Il faut lancer les commande suivante en prenant en compte le temps de synchro, c'est a dire en attendant que la led soit fixe avant de lance pppd.
\note{on est supppos� etre dans le r�prtoire eciadsl cr�� au debut du script}

\begin{verbatim}
$ insmod kernelmode/eci.o
$ ppp/pppd/pppd call test
\end{verbatim}


\end{document}
