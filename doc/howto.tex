% latex for the driver doc
% to compile it:
% make

\documentclass[a4paper,12pt]{article}
\usepackage[francais]{babel}
\usepackage[latin1]{inputenc}
\baselineskip=16pt

\title{\huge{How-to pour le driver Linux du modem ECI USB}}
\author{Bertrand Rougier, Thierry De Baere, Florent Manens, Beno�t Papillault}

\begin{document}
\pagenumbering{arabic}
\maketitle
\newpage
\tableofcontents
\newpage


\section{A qui s'adresse ce document ?}

    Ce document s'adresse � tous les possesseurs du modem
    ECI USB voulant l'utiliser sous Linux.

\section{Configuration requise}

    Assurez vous d'abord que vous poss�dez un noyau Linux 
    de version sup�rieure ou �gale � 2.4.7-10

    Distributions o� le driver fonctionne : 
    \begin {itemize}
    \item   Mandrake Linux 8.1 (noyau 2.4.8-26mk)
    \item   Redhat Linux 7.2 (noyau 2.4.7-10)
    \end{itemize}
    
    Distributions o� le driver ne fonctionne pas :
    \begin {itemize}
    \item   Redhat Linux 7.1 (noyau 2.4.2-2)
    \end{itemize}
    
    Cette liste n'est pas exhaustive. Si vous parvenez � vous connecter
    avec une autre distribution ou avec un noyau plus ancien,
    n'h�sitez pas � nous contacter.

\section {Proc�dure d'installation (� ne faire qu'une seule fois)}
    R�f�rence : archive usermode.tgz pr�sente sur http://flashcode.free.fr/linux

\subsection{Pr�paration du driver}

    Connectez vous en tant que root puis tapez :
    make \\
    make install

    Il arrive parfois que le modem soit allum� quand vous 
    d�marrez votre ordinateur, cela est du au module dabusb,
    il faut l'enlever pour pouvoir installer le driver et faire 
    fonctionner le modem.
    
    Tapez les commandes suivantes en rempla�ant VERSION
    par la version de votre noyau Linux : 
    \footnote{(pour conna�tre la version du noyau tapez : uname -r)}
    
    rm -f /lib/modules/VERSION/kernel/drivers/usb/dabusb.o.gz
    \footnote{( ou rm -f /lib/modules/VERSION/kernel/drivers/usb/dabusb.o selon les distributions)}\\
    depmod -a

    Si vous voulez �viter un reboot :
    \begin {itemize}
    \item Faites lsmod.
    \item Si vous voyez une ligne dabusb : faites rmmod dabusb
    \end{itemize}
    
\subsection{Configuration de votre noyau}

Si vous voulez utiliser l'option 'persist' de pppd afin que votre
connexion se reconnecte automatiquement, alors, il faut patcher votre
noyau ou utiliser un noyau >= 2.4.18-pre3. Vous trouverez le patch
n\_hdlc.c.diff au sein de l'archive des drivers speedtouch :
http://speedtouch.sourceforge.net/. Voici les instructions pour
l'utiliser :

\begin{verbatim}
$ cd /usr/src/linux
$ patch -p1 --dry-run < /repertoire des drivers/n_hdlc.c.diff  ( il y a 2 tirets avant dry-run )
\end{verbatim}

Si aucun message d'erreur n'est renvoy� par la commande patch, tapez
ceci pour effectuer le patch du source :

\begin{verbatim}
$ cd /usr/src/linux
$ patch -p1 < /repertoire des drivers/n_hdlc.c.diff
\end{verbatim}

Voila, compilez ensuite ces modules pour votre noyau :

\begin{verbatim}
$ cd /usr/src/linux
$ make menuconfig
     Character devices --->
     [*] Non-standard serial port support
     <M> HDLC line discipline support
     [*]Unix98 PTY support
$ make clean dep modules
# make modules_install (as root)
\end{verbatim}
    
\subsection{Configuration de pppd pour l'ouverture de la connexion}

\subsubsection{Script /etc/ppp/peers/adsl :}

    Il vient avec l'archive usermode.tgz et se place avec le make install
    Il doit ressembler a �a :
\begin{verbatim}
    # 12/04/2001 Benoit PAPILLAULT <benoit.papillault@free.fr>
    # 08/05/2001 Updated. Added "novj" & removed "kdebug 7"
    #
    # This file could be rename but its place is under /etc/ppp/peers
    # To connect to Internet using this configuration file
    # pppd call adsl, where "adsl" stands for the name of this file

    debug
    kdebug 1
    noipdefault
    defaultroute
    pty "/usr/local/bin/pppoeci -vpi 8 -vci 35"
    sync
    user "adsl@adsl"
    noaccomp
    nopcomp
    noccp
    novj
    holdoff 1
    maxfail 0
    usepeerdns
    noauth
    #lcp-echo-interval 600
    #lcp-echo-failure 10
\end{verbatim}

    \subsubsection{A FAIRE :}
    \begin{tabular}{ll}
    - si vous etes abonne wanadoo :&
    remplacer :\\
        & user "\textit{adsl@adsl}"\\
    &par :\\
        &user "\textit{ fti/votre\_login@fti}"\\

    - si vous etes abonne club-internet :&
    remplacer :\\
        &user "\textit{adsl@adsl}"\\
    &par :\\
        &user "\textit{votre\_login@clubadsl1}"\\
    \end{tabular}
    
    ASTUCE sur certaines machines Linux :
    
        Si vos premieres connexions echoues, voyez dans \textit{ /var/log/messages}
        Si pppd s'arrete sur un message contenant LCP, alors faites cela :\\
        dans \textit{ /etc/ppp/peers/adsl :}
        DE-commentez les 2 derni�res lignes parlant de LCP.

\subsubsection{Script d'authentification pppd}

    Votre password sera stocke dans un script d'authentification
    qui depend de l'operateur internet que vous utilisez :
    
    
    \begin{tabular}{ll}
    Pour wanadoo :&     \textit{ /etc/ppp/chap-secrets}\\
    Pour club-internet :&   \textit{ /etc/ppp/pap-secrets}\\
    \end{tabular}
    
    
    Les 2 scripts ont exactement la meme syntaxe.
    Vous devez creer un ligne comme suit :
    
    
    \begin{tabular}{ll}
    Pour wanadoo :&
        \textit{ fti/votre\_login@fti * votre\_password *}\\

    Pour club-internet :&
        \textit{ votre\_login@clubadsl1 * votre\_password *}\\
    \end{tabular}
    
    
    ATTENTION :\\
    - les "*" sont importants.\\
    - le premier champ de la ligne DOIT etre egal
      au contenu de la ligne user du fichier \textit{ /etc/ppp/peers/adsl}

\section{Proc�dure de connexion}

\subsection{Manuelle (� reproduire � chaque connexion Internet)}

3 possibilit�s de connections suivant votre distribution ou votre hardware :

    Toute distribution et sur Chipset VIA ou INTEL :
    \begin{itemize}
    \item   rmmod usb-uhci
    \item   modprobe usb-uhci
    \item   mount -t usbdevfs none /proc/bus/usb
    \item   startmodem>log
    \end{itemize}
    
    Pour une CM �quip�e d'un chip ALI Aladin :\\
        meme proc�dure que ci-dessus mais remplacez
        les r�f�rences � usb-uhci par usb-ohci

    Si vous avez la Linux Mandrake 8.1, la procedure suivante peut s'appliquer\\
        /etc/rc.d/init.d/usb restart
        startmodem >log

        IMPORTANT (cas particulier sur Mandrake) :\\
        Modification obligatoire du fichier startmodem avec Mandrake :

        \begin {tabular}{ll}
        remplacez la ligne :&
            ppp call adsl\\
        par :&
            pppd call adsl\\
        \end{tabular}

CA NE MARCHE PAS :
    Si vous ne surfez pas apres l'une de ces 3 procedures, voir chapitre 5)

\subsection{Automatique au demarrage de Linux}
    A venir avec les futures versions du driver.
    Inutile tant que le driver est en Beta.

\subsection{Reprise automatique sur coupure}
        A venir avec les futures versions du driver.
        Inutile tant que le driver est en Beta.

\section{problemes connus et solutions}

    A remplir : cela depend de tout le monde :
    decrivez vos problemes et les solutions
    que vous avez trouve sur la ML eci@ml.free.fr
    Nous les incorporerons ICI.

\subsection{La connexion PPP se fait bien, je vois ppp0 dans ifconfig, mais je ne vais pas sur Internet}

    \subsubsection{Action 1 :}
        verifier que vos DNS sont reconnus :
        dans /etc/resolv.conf
        2 lignes sont presentes commencant par nameserver

        A ce jour :
        
%       \begin {tabular}{l} ptete que e v mettre un tableau apres ... on verra
        Chez wanadoo :\\
        nameserver 193.252.19.3\\
        nameserver 193.252.19.4\\

        Chez club-internet :\\
        nameserver 194.117.200.15\\
        nameserver 194.117.200.10\\

        Si ce n'est pas le cas, creer ces 2 lignes dans le fichier.
    
    \subsubsection{Action 2 :}
        faire : nslookup www.wanadoo.fr
        
        si pas de reponse et que ppp0 est toujours la dans ifconfig :\\
        Vous avez surement un probleme de routage :\\
        faire route -n, cela devrait donner une sortie du genre :\\
        
        \begin{tabular}{llllllll}
        Destination  &   Passerelle  &    Genmask     &    Indic& Metric& Ref &   Use& Iface\\
        127.0.0.0   &    0.0.0.0    &     255.0.0.0   &    U  &   0  &    0    &    0 &lo\\
        0.0.0.0    &     212.194.0.1  &   0.0.0.0    &     UG  &  0  &    0    &    0 &ppp0\\
        \end{tabular}
        
        La derniere ligne est importante, c'est la GATEWAY par defaut qui vous fais sortir
        sur Internet...

        Si elle n'y est pas, pas de net, donc vous faite sur le shell:
        
        \textit{ route add default dev ppp0}

        Et ca devrait marcher.

\subsection{pppd s'arrete sur erreur : LCP timeout}

    Cas 1 : vous avez un probleme de timeout sur la connexion PPP :
    
        \textit{Editez etc/ppp/peers/adsl}
    
        DE-Commentez les 2 dernieres lignes :
    
            lcp-echo-interval 600 \\
            lcp-echo-failure 10\par

    Cas 2 : vous avez un probleme d'authentification (PPP ne le dit pas toujours explicitement)

        Corrigez vos scripts :\\
        
            \textit{/etc/ppp/peers/adsl}\\
            et\\
            \textit{/etc/ppp/chap-secrets} (wanadoo) ou \textit{/etc/ppp/pap-secrets} (club-internet)\par

        => retour aux chapitres : 3.2.1 et 3.2.2
    
\subsection{Vous avez ce message d'erreur : Can't find your ECI Telecom USB ADSL Loader}

    \subsubsection{Message d'erreur }
    \begin{verbatim}
    [root@hwi usermode]# ./startmodem
    /proc/bus/usb: No such file or directory
    Can't find your ECI Telecom USB ADSL Loader
    ECI Load 1 : failed!
    /proc/bus/usb: No such file or directory
    Can't find your ECI Telecom USB ADSL WAN Modem
    ECI Load 2 : failed!
    \end{verbatim}
    
    Si vois avez un message qui ressemble a �a, c'est que vous n'avez pas mont� le syst�me de fichier pour l'usb.
    
    \subsubsection{Solution }
    
    \textit{mount -t usbdevfs none /proc/bus/usb}

\subsection{eci-load2 n'arrive pas a avoir la synchronisation // blockage au packet 259}
    \subsubsection{Probl�me}
    
    Le driver n'arrive pas a avoir la synchronisation, il faut alors le relancer.
    
    \subsubsection{Solution}
    
    Pour relancer le driver, nous vous conseillons de repartir du d�but en d�branchant et en rebranchant le modem, il faut ensuite refaire :
    \textit{./startmodem}
    
\subsection{Passer la patate (Debian patato) en kernel 2.4}
    C'est vrai que c'est un peu hors sujet mais �a peux aider ceux qui veulent tester le driver.
    vous trouverez la documentation pour faire ceci a ces adresses : 
    
    http://fs.tum.de/~bunk/kernel-24.html

    Bonne upgrade !

\section{Contacts}

    Pour toute question, abonnez-vous � la liste de diffusion de ce projet 
    en envoyant un mail vide � eci-request@ml.free.fr?subject=subscribe
    et postez ensuite vos remarques ou suggestions � eci@ml.free.fr 
  
\end{document}
