% latex for the driver doc
% to compile it:
% make

% 27/09/2002 FlashCode <flashcode@flashtux.org>
% update this documentation

% 07/02/2002 Benoit PAPILLAULT
% howto add an http link:
% - add the beginning of the document: \usepackage{html}
% - to add a link \htmladdnormallink{text}{URL}

% 15/02/2002 Benoit PAPILLAULT
%   change the name log by log.txt to be easier to send as an attachement
%   to an email. (.txt is automatically recognize).

% CVS $Id$
% Tag $Name$

\documentclass[a4paper,12pt]{article}
\usepackage[francais]{babel}
\usepackage{html}
\usepackage[latin1]{inputenc}
\baselineskip=16pt

\title{\huge{How-to du driver Linux pour modems ECI USB ADSL ou bas\'es sur puce Globespan}}
\author{Bertrand Rougier, Thierry De Baere, Florent Manens, Beno\^{\i}t Papillault, FlashCode}

\begin{document}
\pagenumbering{arabic}
\maketitle
\newpage
\tableofcontents
\newpage


\section{Introduction}

\subsection{A qui s'adresse ce document ?}

Ce document s'adresse \`a tous les possesseurs du modem ADSL ECI USB (ou d'un modem
\'equip\'e d'une puce Globespan) voulant l'utiliser sous Linux.
Vous trouverez la derni\`ere version de ce document sur le site :
\htmladdnormallink{http://eciadsl.flashtux.org}{http://eciadsl.flashtux.org},
section Documentation.

La liste des modems support\'es (ou suppos\'es support\'es) ainsi que la version minimale
du driver pour chaque modem se trouve ici :
\htmladdnormallink{http://eciadsl.flashtux.org/modems.php}{http://eciadsl.flashtux.org/modems.php}

\subsection{Conventions}

Lors des exemples de commandes \`a taper au niveau du shell, les
commandes \`a ex\'ecuter en tant qu'utilisateur normal sont pr\'ec\'ed\'ees de \$
et les commandes \`a ex\'ecuter en root sont pr\'ec\'ed\'ees de \#.

\section{Configuration requise}

%    Assurez vous d'abord que vous poss\'edez un noyau Linux 
%    de version sup\'erieure ou \'egale \`a 2.4.0

\subsection{Distributions}

    Distributions o� le driver fonctionne : 
    \begin{itemize}
    \item   Mandrake Linux 8.1 (noyau 2.4.8-26mk)
    \item   Redhat Linux 7.2 (noyau 2.4.7-10)
    \item   Debian 2.2 ou sup\'erieure (noyau 2.4.0 mimimun)
    \item   Slackware
    \item   etc...
    \end{itemize}
    
    Distributions o� le driver ne fonctionne pas :
    \begin{itemize}
    \item   Redhat Linux 7.1 (noyau 2.4.2-2)
    \end{itemize}
    
    Le driver est en cours de portage vers *Bsd et MacOs X.\\
    Contacter les \htmladdnormallink{d\'eveloppeurs}{http://eciadsl.flashtux.org/support.php}
    pour en savoir plus sur l'avancement de ces travaux.

\subsection{Noyaux}

Pour obtenir la version de votre noyau:
\begin{verbatim}
$ uname -r
\end{verbatim}

    Liste des noyaux test\'es :

    \begin{itemize}
    \item kernel 2.4.0 : ok [usb-uhci]
    \item kernel 2.4.1 : ok [usb-uhci]
    \item kernel 2.4.2 : ok [usb-uhci]
    \item kernel 2.4.3 : ok [usb-uhci]
    \item kernel 2.4.4 : ok [usb-uhci]
    \item kernel 2.4.5 : ok [usb-uhci]
    \item kernel 2.4.6 : ok [usb-uhci]
    \item kernel 2.4.7 : ok [usb-uhci]
    \item kernel 2.4.8 : ok [usb-uhci]
    \item kernel 2.4.9 : ok [usb-uhci]
    \item kernel 2.4.10 : kernel panic process\_iso [usb-uhci] , ok [uhci]
    \item kernel 2.4.12 : ok [usb-uhci]
    \item kernel 2.4.13 : ok [usb-uhci]
    \item kernel 2.4.14 : ok [usb-uhci]
    \item kernel 2.4.15 : ok [usb-uhci]
    \item kernel 2.4.16 : ok [usb-uhci]
    \item kernel 2.4.17 : ok [usb-uhci]
    \item kernel 2.4.18 : ok [usb-uhci]
    \item kernel 2.4.19 : ok [usb-uhci]
    \item kernel 2.5.0 : ok [usb-uhci]
    \end{itemize}

D'autres noyaux sont en cours de test.

\subsection{pppd}

Pour obtenir la version de pppd:
\begin{verbatim}
$ pppd --version
\end{verbatim}

Liste des versions test\'ees:
\begin{itemize}
\item pppd 2.4.0
\item pppd 2.4.1
\end{itemize}

    Cette liste n'est pas exhaustive. Si vous parvenez \`a vous connecter
    avec une autre distribution ou avec un noyau plus ancien,
    n'h\'esitez pas \`a nous contacter.

\section {Proc\'edure d'installation (\`a ne faire qu'une seule fois)}

R\'ef\'erence : archive \textit{usermode-X.Y.tgz} pr\'esente sur
\htmladdnormallink{http://eciadsl.flashtux.org}{http://eciadsl.flashtux.org},
section T\'el\'echargement.

\subsection{Pr\'eparation du driver}

    Connectez vous en tant que root puis tapez :

\begin{verbatim}
$ cd usermode
$ make
# make install
# eciconf.sh     (ou eciconftxt.sh)
\end{verbatim}

    NB: eciconf.sh est un utilitaire de configuration graphique du driver.
    Il doit par cons\'equent \^etre lanc\'e sous X-window et n\'ecessite la pr\'esence
    des librairies Tcl/Tk.
    
    Si vous n'avez pas X ou Tcl/Tk install�, vous pouvez lancer eciconftxt.sh
    (meme outil qu'eciconf.sh mais en mode texte pour console / terminal).

    Il arrive parfois que le modem soit allum\'e quand vous 
    d\'emarrez votre ordinateur, cela est d� au module dabusb,
    il faut l'enlever pour pouvoir installer le driver et faire 
    fonctionner le modem.
    
    3 possibilit\'es pour enlever ce module :
    
    \textbf{1)} Par eciconf.sh : d\'ebranchez votre modem puis dans eciconf.sh cliquez
    sur "Remove dabusb module".
    
    \textbf{2)} Par eciconftxt.sh : choix 2, puis suivez les instructions.
    
    \textbf{3)} Tapez les commandes suivantes en rempla�ant VERSION par la version de
    votre noyau Linux : 
    \footnote{(pour conna\^{\i}tre la version du noyau tapez : uname -r)}
    
    rm -f /lib/modules/VERSION/kernel/drivers/usb/dabusb.o.gz
    \footnote{( ou rm -f /lib/modules/VERSION/kernel/drivers/usb/dabusb.o selon les distributions)}\\
    depmod -a

    Si vous voulez \'eviter un red\'emarrage :
    \begin {itemize}
    \item Faites lsmod.
    \item Si vous voyez une ligne dabusb : faites rmmod dabusb
    \end{itemize}

\subsection{ Configuration de votre noyau (optionnel) }

Si vous voulez utiliser l'option 'persist' de pppd afin que votre
connexion se reconnecte automatiquement, alors il faut patcher votre
noyau ou utiliser un noyau >= 2.4.18-pre3. Vous trouverez le patch
\textit{n\_hdlc.c.diff} au sein de l'archive des drivers Speedtouch :
\htmladdnormallink{http://speedtouch.sourceforge.net/}{http://speedtouch.sourceforge.net/}.
Voici les instructions pour l'utiliser :

\begin{verbatim}
$ cd /usr/src/linux
$ patch -p1 --dry-run < /r\'epertoire des drivers/n_hdlc.c.diff  ( il y a 2 tirets avant dry-run )
\end{verbatim}

Si aucun message d'erreur n'est renvoy\'e par la commande patch, tapez
ceci pour effectuer le patch du source :

\begin{verbatim}
$ cd /usr/src/linux
$ patch -p1 < /r\'epertoire des drivers/n_hdlc.c.diff
\end{verbatim}

Voila, compilez ensuite ces modules pour votre noyau :

\begin{verbatim}
$ cd /usr/src/linux
$ make menuconfig
     Character devices --->
     [*] Non-standard serial port support
     <M> HDLC line discipline support
     [*]Unix98 PTY support
$ make clean dep modules
# make modules_install (as root)
\end{verbatim}
    
\subsection{Configuration de pppd pour l'ouverture de la connexion}

Si vous avez utilis\'e eciconf.sh, vous pouvez aller directement \`a la section 4.

\subsubsection{Script /etc/ppp/peers/adsl :}
\label{cha:peers}

Il vient avec l'archive \textit{usermode-X.Y.tgz} et se place avec le make
install. Il doit ressembler \`a �a :

\begin{verbatim}
# 12/04/2001 Benoit PAPILLAULT <benoit.papillault@free.fr>
# 08/05/2001 Updated. Added "novj" & removed "kdebug 7"
# 07/02/2002 Replace "maxfail 0" by "maxfail 10"
# 29/04/2002 Added the option "linkname" to easily locate the running pppd
#
# This file could be rename but its place is under /etc/ppp/peers
# To connect to Internet using this configuration file, type
# pppd call adsl, where "adsl" stands for the name of this file

debug
kdebug 1
noipdefault
defaultroute
pty "/usr/local/bin/pppoeci -v 1 -vpi 8 -vci 35 -vendor 0x0915 -product 0x8000"
sync
user "adsl@adsl"
noaccomp
nopcomp
noccp
novj
holdoff 10

# This will store the pid of pppd in the first line of /var/run/ppp-eciadsl.pid
# and the interface created (like ppp0) on the second line.
linkname eciadsl

# maxfail is the number of times pppd retries to execute pppoeci after
# an error. If you put 0, pppd retries forever, filling up the process table
# and thus, making the computer unusable.
maxfail 10

usepeerdns
noauth

# If your PPP peer answer to LCP EchoReq (lcp-echo requests), you can
# use the lcp-echo-failure to detect disconnected links with:
#
# lcp-echo-interval 600
# lcp-echo-failure 10
#
# However, if your PPP peer DOES NOT answer to lcp-echo request, you MUST
# desactivate this feature with the folowing line
#
lcp-echo-interval 0

# You may need the following, but ONLY as a workaround
# mtu 1432
\end{verbatim}

\subsubsection{A FAIRE :}
\label{cha:faire}

    \begin{tabular}{ll}
    - si vous \^etes abonn\'e wanadoo :&
    remplacer :\\
        & user "\textit{adsl@adsl}"\\
    &par :\\
        &user "\textit{ fti/votre\_login@fti}"\\

    - si vous \^etes abonn\'e club-internet :&
    remplacer :\\
        &user "\textit{adsl@adsl}"\\
    &par :\\
        &user "\textit{votre\_login@clubadsl1}"\\
    \end{tabular}
    
    ASTUCE sur certaines machines Linux :
    
        Si vos premi\`eres connexions \'echouent, voyez dans \textit{ /var/log/messages}
        Si pppd s'arr\^ete sur un message contenant LCP, alors faites cela :\\
        dans \textit{ /etc/ppp/peers/adsl :}
        DE-commentez les 2 derni\`eres lignes parlant de LCP.

\subsubsection{Script d'authentification pppd}

    Votre mot de passe sera stock\'e dans un script d'authentification
    qui d\'epend de l'op\'erateur internet que vous utilisez :
    
    
    \begin{tabular}{ll}
      Pour wanadoo :&     \textit{ /etc/ppp/chap-secrets}\\
      Pour club-internet :&   \textit{ /etc/ppp/pap-secrets}\\
    \end{tabular}
    
    
    Les 2 scripts ont exactement la m\^eme syntaxe.
    Vous devez cr\'eer une ligne comme suit :
    
    
    \begin{tabular}{ll}
Pour wanadoo       :& \textit{ fti/votre\_login@fti * votre\_password *}\\
Pour club-internet :& \textit{ votre\_login@clubadsl1 * votre\_password *}\\
    \end{tabular}
    
    
    ATTENTION :\\
    - les "*" sont importants.\\
    - le premier champ de la ligne DOIT \^etre \'egal
      au contenu de la ligne user du fichier \textit{ /etc/ppp/peers/adsl}

\section{Proc\'edure de connexion}

\subsection{Manuelle (\`a reproduire \`a chaque connexion Internet)}

C'est tr\`es simple, le script startmodem s'occupe de tout ou presque:

\begin{verbatim}
$ /usr/local/bin/startmodem | tee log.txt
\end{verbatim}

�A NE MARCHE PAS : Si vous ne surfez pas apr\`es l'une de ces 3
proc\'edures, voir chapitre \ref{cha:problemes}.

\subsection{Automatique au d\'emarrage de Linux}

A venir avec les futures versions du driver.  Inutile tant que le
driver est en B\^eta.

\subsection{Reprise automatique sur coupure}

Le driver \'etant pour l'instant en version b\^eta, cette partie ne
fonctionne pas de mani\`ere stable. N\'eanmoins, voici la proc\'edure \`a
suivre. Tout d'abord, assurez vous que le module HDLC n'est pas bugg\'e,
vous pouvez ais\'ement faire cette v\'erification avec
\textit{eci-doctor.sh}. Ensuite, ajouter l'option \textbf{persist} au
fichier \textit{/etc/ppp/peers/adsl}. Relancer pppd et normalement,
votre connexion devrait \^etre op\'erationnelle 24h/24.

\section{Probl\`emes connus et solutions}
\label{cha:problemes}

Le driver est livr\'e avec un utilitaire de diagnostic des erreurs de
configuration les plus courantes. Pour l'utiliser, allez dans le
r\'epertoire \textit{usermode} de l'archive des drivers et lancez
\textit{./eci-doctor.sh}.

\begin{verbatim}
# cd usermode
# ./eci-doctor.sh
\end{verbatim}

Si tout est correct, vous aurez les messages suivants:

\begin{verbatim}
Support for USB is OK
Preliminary USB device filesystem is OK
UHCI support is OK
/dev/ppp is OK
HDLC support is OK
HDLC support is OK (no bug)
/etc/ppp/chap-secrets is OK
PPP connection is OK
Default route over ppp0 is OK
Everything is OK
\end{verbatim}

Avant de contacter la liste de diffusion ou qui que ce soit, lisez ce
qui suit. Sinon, n'oubliez pas de joindre le fichier \textit{log.txt}
obtenu lors du lancement de startmodem.

A remplir : cela d\'epend de tout le monde : d\'ecrivez vos probl\`emes et
les solutions que vous avez trouv\'ees sur la liste de diffusion
eci@ml.free.fr Nous les incorporerons ICI.

\subsection{Eci-load2 n'arrive pas a avoir la synchronisation // blocage au paquet 259 (ou autre paquet)}

Il peut arriver que le driver n'arrive pas \`a avoir la synchronisation.
Il faut alors le relancer (nous vous conseillons de repartir du
d\'ebut en d\'ebranchant et en rebranchant le modem, il faut ensuite refaire : \textit{./startmodem})

Par contre si le blocage est syst\'ematiquement sur le m\^eme paquet apr\`es plusieurs tentatives,\\
il faut changer le fichier .bin qui sert \`a synchroniser le modem.
Des fichiers .bin "tout pr\^ets" sont disponibles ici :
\htmladdnormallink{http://eciadsl.flashtux.org/download.php#sync}{http://eciadsl.flashtux.org/download.php#sync}
Si aucun .bin ne fonctionne, vous devez alors cr\'eer votre propre .bin
(voir sur la page ci-dessus la documentation)

Il y a 2 fa�ons de changer le .bin :

    \subsubsection{Fa�on 1 :}
        Avec une version du driver strictement sup\'erieure \`a 0.5 (CVS ou >= 0.6) :\\
        le changement se fait dans eciconftxt.sh
    
    \subsubsection{Fa�on 2 :}
        Avec toutes les versions du driver, vous pouvez effectuer ceci :\\
        \'ecraser le fichier /etc/eciadsl/eci_wan3.bin par un autre .bin\\
        
        Exemple :
\begin{verbatim}
cp /mon_chemin/eci_wan3.bordeaux.bin /etc/eciadsl/eci_wan3.bin
\end{verbatim}

\subsection{La connexion PPP se fait bien, je vois ppp0 dans ifconfig, mais je ne vais pas sur Internet}

    \subsubsection{Action 1 :}
        v\'erifier que vos DNS sont reconnus :
        dans \textit{/etc/resolv.conf}
        2 lignes sont pr\'esentes commen�ant par nameserver

        A ce jour :
        
%       \begin {tabular}{l} ptete que e v mettre un tableau apr\`es ... on verra
        Chez wanadoo :\\
\begin{verbatim}
nameserver 193.252.19.3
nameserver 193.252.19.4
\end{verbatim}

        Chez Club Internet :\\
\begin{verbatim}
nameserver 194.117.200.15
nameserver 194.117.200.10
\end{verbatim}

        Chez 9 Telecom :\\
\begin{verbatim}
nameserver 212.30.96.108
nameserver 213.203.124.146
\end{verbatim}

        Chez Cegetel :\\
\begin{verbatim}
nameserver 194.6.128.3
nameserver 194.6.128.4
\end{verbatim}

        Chez World Online :\\
\begin{verbatim}
nameserver 212.83.128.3
nameserver 212.83.128.4
\end{verbatim}

        Chez Telecom Italia :\\
\begin{verbatim}
nameserver 212.216.112.112
nameserver 212.216.172.62
\end{verbatim}

        Chez Tiscali Italia :\\
\begin{verbatim}
nameserver 195.130.224.18
nameserver 195.130.225.129
\end{verbatim}

        Chez Pipex UK :\\
\begin{verbatim}
nameserver 158.43.240.4
nameserver 158.43.240.3
\end{verbatim}

        Si ce n'est pas le cas, cr\'eer ces 2 lignes dans le fichier.
    
    \subsubsection{Action 2 :}
        faire : nslookup www.wanadoo.fr
        
        si pas de r\'eponse et que ppp0 est toujours la dans ifconfig :\\
        Vous avez s�rement un probl\`eme de routage :\\
        faire route -n, cela devrait donner une sortie du genre :\\
        
        \begin{tabular}{llllllll}
        Destination  &   Passerelle  &    Genmask     &    Indic& Metric& Ref &   Use& Iface\\
        127.0.0.0   &    0.0.0.0    &     255.0.0.0   &    U  &   0  &    0    &    0 &lo\\
        0.0.0.0    &     212.194.0.1  &   0.0.0.0    &     UG  &  0  &    0    &    0 &ppp0\\
        \end{tabular}
        
        La derni\`ere ligne est importante, c'est la GATEWAY par d\'efaut qui vous fais sortir
        sur Internet...

        Si elle n'y est pas, pas de net, donc vous faites sur le shell :
\begin{verbatim}
# route add default dev ppp0
\end{verbatim}

        Et �a devrait marcher.

\subsection{pppd s'arr\^ete sur erreur : LCP timeout}

    Cas 1 : vous avez un probl\`eme de timeout sur la connexion PPP :
    
        \textit{�ditez etc/ppp/peers/adsl}
    
        DE-Commentez les 2 derni\`eres lignes :
\begin{verbatim}
lcp-echo-interval 600
lcp-echo-failure 10
\end{verbatim}

    Cas 2 : vous avez un probl\`eme d'authentification (PPP ne le dit pas toujours explicitement)

        Corrigez vos scripts :\\
        
            \textit{/etc/ppp/peers/adsl}\\
            et\\
            \textit{/etc/ppp/chap-secrets} (wanadoo) ou \textit{/etc/ppp/pap-secrets} (club-internet)\par

        => retour aux chapitres : \ref{cha:peers} et \ref{cha:faire}
    
\subsection{Vous avez ce message d'erreur : Can't find your ECI Telecom USB ADSL Loader}

    \subsubsection{Message d'erreur }
    \begin{verbatim}
    [root@hwi usermode]# ./startmodem
    /proc/bus/usb: No such file or directory
    Can't find your ECI Telecom USB ADSL Loader
    ECI Load 1 : failed!
    /proc/bus/usb: No such file or directory
    Can't find your ECI Telecom USB ADSL WAN Modem
    ECI Load 2 : failed!
    \end{verbatim}
    
    Si vous avez un message qui ressemble a �a, c'est que vous n'avez pas mont\'e le syst\`eme de fichier pour l'usb.
    
    \subsubsection{Solution }
\begin{verbatim}
# mount -t usbdevfs none /proc/bus/usb
\end{verbatim}

\subsection{Messages "Timeout USB" une fois connect\'e}

    Ajoutez l'option "mtu 1432" dans votre script /etc/ppp/peers/adsl

\section{Contacts}

\subsection{Support}

Tous les moyens de faire appel au support sont ici :
\htmladdnormallink{http://eciadsl.flashtux.org/support.php}{http://eciadsl.flashtux.org/support.php}

\subsection{Liste de diffusion}

Pour toute question, abonnez vous \`a la liste de diffusion de ce projet
en envoyant un mail vide \`a
\htmladdnormallink{\textbf{eci-request@ml.free.fr}}{mailto:eci-request@ml.free.fr?subject=subscribe}
avec comme sujet \textbf{subscribe} et postez ensuite vos remarques ou
suggestions \`a
\htmladdnormallink{\textbf{eci@ml.free.fr}}{mailto:eci@ml.free.fr}.

\subsection{IRC}

Vous pouvez vous connecter sur le r\'eseau
\htmladdnormallink{OpenProjects.Net}{http://www.openprojects.net/} sur
le channel \textbf{\#eci} en utilisant le serveur
\htmladdnormallink{\textbf{irc.openprojects.net}}{irc://irc.openprojects.net/eci}
par exemple.
  
\end{document}
